\chapter{Conclusions and future work}
The following project is the beginning of the $\Delta$Q oscilloscope, our initial goal was to create an application to observe running distributed applications, namely, Erlang ones.
    A prototype was successfully created thanks to the following feats:
    \begin{itemize}
        \item The graphical dashboard for the $\Delta$Q oscilloscope, built in C++, which allows real time display of $\Delta$Qs for the probes inserted in the system.
        \item Fast convolution algorithms to perform statistical analysis on probes.
        \item The creation of a textual syntax to create outcome diagrams.
        \item The \texttt{dqsd\_otel} Erlang wrapper to connect an OpenTelemetry instrumented Erlang app to the oscilloscope.
    \end{itemize}

    The user has full control over the outcome diagrams and can update them dynamically to add or remove probes, this allows full control of what the user decides to include, allowing a finer grained representation or a more general view of the system.  

    The oscilloscope and the Erlang can communicate via TCP socket connections to exchange outcome instances and probe parameters,

    We showed how it can be useful in detecting early signs of overload many crucial features are still missing from the dashboard, and it could require less code modifications in the Erlang side. The next important step of the oscilloscope is its trial in a true distributed application. This would further reinforce the solidity of the paradigm in detecting problems in design of large systems. 

    \section{Future improvements}
        We believe the oscilloscope and the Erlang application can be drastically improved, the size of the project and its intended goal is too big to be encompassed in a single master thesis. We list here some improvements which could be made to both the oscilloscope and the wrapper.
        \subsection{Oscilloscope improvements}
           \begin{itemize}
            
                \item The oscilloscope could be turned into a \textbf{web app}, we feel that a C++ oscilloscope is a good prototype and proof of concept, but its usability would be greater in a browser context. It would be great as a plugin for already existing observability platforms like Grafana.
            
                \item A wider selection of \textbf{triggers}, as of writing this thesis, only the QTA trigger and load are available, this is a limitation due to time constraints. Nevertheless, triggers can be easily implemented in the available codebase.
            
                \item \textbf{Better communication between stub - server - oscilloscope}. The current way of sending outcome instances may be a limiting factor under high load, if hundred of thousands of spans were to be sent, the current way the server and oscilloscope are tied together may throttle communications. TCP socket connections could quickly become the chokepoint which makes the oscilloscope temporarily unusable.

            Future improvements on the server side could implement epoll system server calls to make the server more efficient; \textbf{Detaching server from client}, as of right now, the oscilloscope and the server are tied together, using ZeroMQ to assure real time server-client communications could be an interesting solution to explore.

                \item \textbf{Improve real time graphs}. The class QtCharts does not perform correctly with high frequencies update. Moreover, since we are plotting multiple series (from a minimum 4 to a maximum of 9) per probe, which allows up to 1000 bins per probe, the performance quickly degrades with more probes being displayed. A better graphing class for Qt could definitely improve the experience.

                \item \textbf{Saving probe parameters}: As of writing this thesis, there is no way to save the parameters one may have set. 
           
                \item \textbf{Deconvolution}: An important aspect of $\Delta$QSD, which was not introduced in this paper is deconvolution. It is used to check for infeasability in system desing. Since convolution has already been implemented, this could be integrated using the FFTW3 library. 

                \item \textbf{Exporting graphs}: The graphs can only be observed in the oscilloscope and have no way to be exported to other programs via standard formats.

                \item \textbf{Many more}: This oscilloscope is just a start, if we were to list everything we may want to add, it would take many pages. What we provide is a sufficient enough basis to provide possibilities to observe a running system and understand the power of $\Delta$QSD in analysing its behaviour.
           \end{itemize} 
                
      \subsection{Wrapper improvements}

        \begin{itemize}
            \item As suggested by Bryan Naegele, a member of the observability group of Erlang, the wrapper, instead of working on top of OpenTelemetry, could be directly included inside the context of a span by using the ctx library \cite{ctx}, which provides deadlines for contexts, propagating the value in otel\_ctx, making it available to the OpenTelemetry span processor. Leveraging \texttt{erlang:send\_after} as we already do, we could create outcome instances with telemetry events to handle successful executions and timeouts. The span processor will then be responsible for creating outcome instances, without creating the need for custom functions in the wrapper, like we have now.
        \end{itemize}
   
    \subsection{Real applications}
        A flaw of the oscilloscope and wrapper is that they have not been tested on real applications, while their usefulness has been proven on synthetic applications, the lack of real life applications is a weakness.

    \subsection{Licensing limitations}
    Lastly, a notable limitation is created by \textbf{Qt}, namely, QtCharts. The usage of Qt does not allow us to release our project under BSD/MIT licenses, but rather a GPLv3 one (we cannot release it under LGPL due to QtCharts). \cite{qt-gpl}
