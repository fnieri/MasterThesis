\chapter*{Abstract}
    It is difficult to study the detailed behaviour of large distributed systems while they are running. What happens when there is an overload? How can we feel something is wrong with the system before anything problematic can be observed? Current observability tools do not meet observability requirements when it comes to detecting problems early enough in running systems. 
    
    This thesis aim to provide further proof about how the $\Delta$QSD paradigm can be used to study the behaviour of running systems and to explore tradeoffs in system design, thanks to the implementation of the \textbf{$\Delta$Q oscilloscope}, a real time graphical dashboard that gives insights into a running Erlang system. The development of an Erlang wrapper, named \texttt{dqsd\_otel}, allows the running system to communicate with the oscilloscope to receive real time insights about the execution of the former.
    
    The oscilloscope performs statistical computations on the time series data it receives and displays the results in real time, thanks to the $\Delta$QSD paradigm. We provide a set of triggers which are set to capture rare events, like an oscilloscope would, and give a snapshot of the system under observation as if it was frozen in time. An implementation of a syntax to create outcome diagrams allows the creation of outcome diagrams which give an "observational view" of the system. Furthermore, the implementation of efficient algorithms allows for the computations to be done rapidly on precise representations of components.

    We first provide an extensive summary of $\Delta$QSD concepts, which have been extended to allow the instrumentation of Erlang systems. Subsequently, we explain the user level concepts which are essential to understand how  the oscilloscope works and understand what is displayed on the screen, delving later on into the mathematical foundations of the concepts. Lastly, we provide synthetic applications which prove the soundness of $\Delta$QSD and show how the oscilloscope is able to detect problems in a running system, diagnose it and explore design tradeoffs.



