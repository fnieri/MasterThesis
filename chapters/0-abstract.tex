\chapter*{Abstract}
    It is difficult to study the detailed behaviour of large distributed systems while they are running. What happens when there is an overload? Modern software development practices successfully fail to adequately consider essential quality requirements or even to consider properly whether a system can actually meet its intended outcomes. Current observability tools do not meet observability requirements when it comes to detecting problems early enough in running systems. 

    To tackle these problems, PNSol Ltd. developed the $\Delta$QSD paradigm, a novel metrics-based and quality-centric paradigm that uses formalised outcome diagrams to explore the performance consequences of design decisions, as a performance blueprint of the system. PNSol Ltd. analyses the behaviour of existing systems using $\Delta$QSD, but the technology only works a posteriori, there is no way yet to analyse a system’s behaviour in real time. 

    To advance the usage of $\Delta$QSD in distributed projects, this project aims to develop a prototype of the $\Delta$Q oscilloscope, a dashboard to observe a running Erlang system in real time. The oscilloscope works by communicating to a wrapper which attaches probes to an Erlang system, the "system under test", which translates OpenTelemetry spans to outcome instances and sends them to the oscilloscope. 

In order to use the $\Delta$Q Oscilloscope, the first step is to determine the system's outcome diagram. The system's outcome diagram is a directed graph that shows the causal relationships between the system's internal operations (called outcomes) and its overall inputs and outputs. Each outcome corresponds to a system component. It has a start event and stop event, when the component is called and when it returns.

For each outcome of interest, an observation point (probe) is attached to measure the delay of that outcome. As many probes are put into the system as are needed to observe the desired system behaviours. The probes are implemented on top of OpenTelemetry tracing ability. Each probe sends a time series of data to the $\Delta$Q Oscilloscope, which performs statistical computations on all the time series and displays the results in real time.

The oscilloscope provides a system of triggers, which, like a true oscilloscope, captures rare events in a distributed system. The oscilloscope can provide snapshots of the running system under test which provide a still of the system, as if it was frozen in time.
