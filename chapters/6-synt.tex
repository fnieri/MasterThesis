\chapter{Evaluation on synthetic programs}
    This chapter evaluates the usefulness of the oscilloscope by testing it on three distinct synthetic Erlang applications which can be found in \cref{app:synte}. Each application was first represented by an outcome diagram in the oscilloscope. It was then instrumented with the adapter to communicate to the oscilloscope.
    Although these use cases may be simple, we show that the oscilloscope can be a powerful tool in detecting non-linear behaviour with microseconds precision. The examples are:
    \begin{itemize}
        \item A system with sequentially composed outcomes. It shows how non-linear behaviour can be detected in the oscilloscope at the microseconds level. The example leverages M/M/1/K queues to show how typical queue behaviour is represented on the oscilloscope.
        \item Then, we provide two applications that perform synchronisation between components. For this application we use two different operators: first-to-finish and all-to-finish. There, we show how these operators can aid in detecting slower components in the system.
    \end{itemize}
    \section{System with sequential composition}
    We model a first system with two sequentially composed component. We choose two model the two components as M/M/1/K queues. 
   
   \paragraph{Why M/M/1/K queues?} An average component in a distributed system can be modeled as an M/M/1/K, due to the exponential inter-arrival rate of messages $\lambda$, the exponential distribution of the execution delay $\mu$, the buffer size of messages $K$ of a component and the failure rate $f$. \cite{dq-tut}
    
    Let us first provide a refresher about M/M/1/K queues:
    \begin{itemize}
        \item $\lambda$: The arrival rate.
        \item s: The service time, is the time it takes to serve a message.
        \item $\mu$: The service \textbf{rate} and E[s] = $\dfrac{1}{\mu}$
        \item Offered load: $\rho = \dfrac{\lambda}{\mu}$
    \end{itemize}

    We will control $\lambda$ to show its effects on the offered load. The offered load can tell much about the system:
    \begin{itemize}
        \item At low load ($\rho < 0.8$) the failure will tend to 0, the system is behaving correctly and the $\Delta$Q will show that, the delay will tend to 1.
        \item Once $\rho$ is approaching high load ($\rho > 0.8$) we can observe the failure increasing quickly, but we can observe the system starting to get bad after $\rho > 0.5$! \cite{dq-tut}
    \end{itemize}
    
    \subsection{System composition}
    The system has two components \texttt{worker\_1}, \texttt{worker\_2}. Each individual component is composed of a queue of size $K = 1000$ and a worker process.
    
    The system sends $n$ messages per second following a Poisson distribution to \texttt{worker\_1}'s queue. 
    The queue notifies its worker, which then does $N$ loops, which are defined upon start, of fictional work. Once done, worker\_1 then passes a message to \texttt{worker\_2}'s queue, which has another queue of same size, who passes the message to \texttt{worker\_2}'s worker, which does the same amount of loops as worker\_1. When a worker completes its work, it notifies its queue, freeing one message from its buffer size and allowing the next message to be processed.
    
    If the queue's buffer is overloaded, it will drop the incoming message and consider the execution a failure.
    
    A probe named ``$probe$'' is defined, which observes the execution from when the message is sent to \texttt{worker\_1} up until \texttt{worker\_2} is done.

    Lastly, both workers share the same processor, to observe potential non-linearity.
    \begin{figure}[H]
        \begin{center}
            \includegraphics[scale=1.2, width=\textwidth]{tikz/mm1k.pdf} 
        \end{center}
        \caption{Outcome diagram of the M/M/1/K queue with the label coloured lines representing the probes that were inserted.}
        \label{fig:mm1k}
    \end{figure}

    \subsection{Determining parameters dynamically}
        We stated previously that determining parameters is something that must be done with an underlying knowledge of the system. The oscilloscope can provide knowledge of the system, here is an example of worker\_1 and worker\_2 as observed in the oscilloscope.

        The engineer supposes that the workers executions should take a maximum of 4 ms to complete, but doesn't actually know how long the executions should take. The engineer, after having set the required parameters observes in the following graph in the oscilloscope \cref{fig:w1w2}.

    The oscilloscope shows the engineer that their assumptions do not correspond to the actual system $\Delta$Q, the user can then modify the parameters to observe the actual system's behaviour. By setting $dMax$ to 8 ms, they can observe the worker's $\Delta$Qs failure approaching 0.

    On the other hand, the engineer's assumption could have been what they truly expected from the system, in this case, the oscilloscope tells him that the system is not behaving as expected. 
\begin{figure}[H]
            \centering
            \begin{subfigure}{.5\textwidth}
                \centering
                \includegraphics[width=0.98\textwidth]{img/overload_2/worker_1a.png}
                \label{fig:w14}
            \end{subfigure}%
            \begin{subfigure}{.5\textwidth}
                \centering
                \includegraphics[width =0.98\textwidth]{img/overload_2/worker_1_8.png}
                \label{fig:w18}
            \end{subfigure}
            \label{fig:w1w2hb}
            \caption{Left: worker\_1 $\Delta$Q with $dMax = 4ms$. (Green, between the arrow): Mean and confidence bounds. (1, blue): Observed $\Delta$Q. The worker failure tends to 0.25. \\
            Right: worker\_1 $\Delta$Q with $dMax$ = 8ms. The worker failure now tends to 0. \\
            In both graphs we can observe how the observed $\Delta$Q of the sampling window is not precise. The step function representation of it fluctuates. The mean and confidence bounds provide a more precise representation of the $\Delta$Qs of worker\_1 over a polling window.}%
            \end{figure}
    \subsubsection{Low Load} 
    Let's first observe the system at low load, we will send 50 messages per second to observe the system under test to get key properties.
    We can observe that the average worker's execution takes $\approx$ 30ms. We then have $\mu_{worker} = \dfrac{1}{0.0033} \approx 300$ req/s. Thus, $\rho = \dfrac{50}{322} = 0.16$, we are in nice grounds!

    At low load, the system is behaving \textbf{linearly}. We can observe in the oscilloscope the probe \textbf{observed $\Delta$Q} and \textbf{calculated $\Delta$Q} confidence bounds overlap. 
        \begin{figure}[H]
            \centering
            \begin{subfigure}{.5\textwidth}
                \centering
                \includegraphics[width=0.98\textwidth]{img/overload_2/50_workeran.png}
                \label{fig:norm_ex_1}
            \end{subfigure}%
            \begin{subfigure}{.5\textwidth}
                \centering
                \includegraphics[width =0.98\textwidth]{img/overload_2/50_probe2.png}
                \label{fig:norm_ex_2}
            \end{subfigure}
            \label{fig:norm_ex}
            \caption{Graphs for $\lambda = 50$. Left: worker\_1 $\Delta$Qs. (1, Blue): The observed $\Delta$Q. (Green, between arrows): The confidence bounds. \\
            Right: ``probe'' $\Delta$Qs. (1, blue): The observed $\Delta$Q. (2, red): The calculated $\Delta$Q. (Magenta, between arrows): the observed and calculated $\Delta$Qs confidence bounds overlapping.}
        \end{figure}
    
\subsubsection{Early signs of overload}
    
    At load = 0.5 the system should start showing dependent behaviour. Let us observe what happens when $\lambda = 150 \rightarrow \rho = 0.5$.
    
    Recall \ref{fig:cdf_indep}, a linear system should obey superposition, and the sum of the delay of the workers should be equal to the overall delay of ``probe''. This is not the case! \\
    We can start to observe early signs of dependency! At load 0.5 the calculated $\Delta$Q is deviating from the observed one. This is a sign that the performance is degrading. At the 50th percentile the deviation is minimal, around 0.4 ms. Nevertheless, the precision of the paradigm allows even for these minimal deviations to appear on the graphs, being able to recognise early signs of overload approaching. \\
    The superposition principle is not respected anymore, there is apparent \textbf{non-linearity}, and the fact that $\Delta$Q can recognise non-linear behaviour with \textbf{0.4ms precision} is impressive. The oscilloscope is a very powerful tool that can help assess dependent behaviour in running systems!
 
       \begin{figure}[H]
            \centering
            \begin{subfigure}{.5\textwidth}
                \centering
                \includegraphics[width=0.98\textwidth]{img/overload_2/150_probe2.pdf}
                \label{fig:ovuvv}
            \end{subfigure}%
            \begin{subfigure}{.5\textwidth}
                \centering
                \includegraphics[width =0.98\textwidth]{img/overload_2/150_probe2zoom_cropped.pdf}
                \label{fig:ovovv}
            \end{subfigure}
            \label{fig:avavv} 
        \end{figure}
       
\subsubsection{High load}
    Performance degradation at 0.5 offered load is already remarkable, the $\Delta$Qs even if the difference is seemingly minimal. We can go even further and observe the system under high load situations. We set $\lambda = 200 \rightarrow \rho = 0.83$, just above the high load threshold.
    
       \begin{figure}[H]
            \centering
            \begin{subfigure}{.5\textwidth}
                \centering
                \includegraphics[width=0.98\textwidth]{img/overload_2/250_worker.png}
                \label{fig:high_load_1}
            \end{subfigure}%
            \begin{subfigure}{.5\textwidth}
                \centering
                \includegraphics[width =0.98\textwidth]{img/overload_2/250_probe2.png}
                \label{fig:high_load_2}
            \end{subfigure}
            \label{fig:early_ov}
            \caption{Graphs for $\lambda = 250$. Left: worker\_1 $\Delta$Q with confidence bounds. \\
            Right: probe $\Delta$Q, in blue (1), the observed $\Delta$Q with its confidence bounds, in red (2) the calculated $\Delta$Q with its confidence bounds.}
        \end{figure}
    
    This is what we expected previously, and confirms what is expected by queueing theory, $\Delta$Q is capable of observing the basic observation requirements and capable of recognising dependency. While what is expected by the execution of the queue (observed $\Delta$Q) is a nice normally distributed CDF with little to no failure. What we can actually observe is a degraded performance in both workers and the probe observed execution.

    The workers CDF has completely degraded, with the average request taking almost double the time as under normal queueing conditions. 

    Further degradation can be observed by increasing $\lambda = {300, 350} \rightarrow \rho \ge 1$.

    \begin{figure}[H]
            \centering
            \begin{subfigure}{.5\textwidth}
                \centering
                \includegraphics[width=0.98\textwidth]{img/overload_2/300_probe2.png}
                \label{fig:high_load_1}
            \end{subfigure}%
            \begin{subfigure}{.5\textwidth}
                \centering
                \includegraphics[width =0.98\textwidth]{img/overload_2/350_probe2.png}
                \label{fig:high_load_2}
            \end{subfigure}
            \label{fig:early_ov}
            \caption{Left: (1) probe calculated $\Delta$Q and confidence bounds. (2) probe observed $\Delta$Q and confidence bounds at $\lambda = 300$. Right: probe $\Delta$Qs at $\lambda = 350$}
        \end{figure}

    The system degrading clear, the $\Delta$Qs show how almost all messages are being dropped or take $> dMax$. Let us look at triggers and how they can be useful to diagnose such cases.
  
    \subsubsection{Triggers}
        By observing the system under test in high load cases, we can set the load trigger by setting the sampling window to 1 second and trigger when outcome instances $\gtrapprox 150$. We can also set a trigger based on observation of the running system.

        \paragraph{QTA trigger}
            By observing the system, we create a QTA for the probe with: 25\% = 0.0075 s, 50\% = 0.0125 s, 75\% = 0.015s and minimum intangible mass = 0.9.

            By setting the trigger to fire for $\Delta$Q$_{obs} < QTA$. We captured a handful of snapshots. Here, $\lambda = 150$.
        
        \begin{figure}[H]
            \centering
            \begin{subfigure}{.5\textwidth}
                \centering
                \includegraphics[width=0.98\textwidth]{img/overload_2/snapshots.png}
                \label{fig:high_load_1}
            \end{subfigure}%
            \begin{subfigure}{.5\textwidth}
                \centering
                \includegraphics[width =0.98\textwidth]{img/overload_2/qta_triggerd2.png}
                \label{fig:high_load_2}
            \end{subfigure}
            \label{fig:early_ov}
            \caption{Left: Snapshots for fired triggers. \\
            Right: (1) Observed $\Delta$Q violating QTA (3). The confidence bounds for the observed $\Delta$Q (blue, arrow, below) shows the system deteriorating. The calculated $\Delta$Q (2) is behaving worse than its confidence bounds (red, arrow, above).} % 
        \end{figure}

    QTA triggers can help to detect overhead even before high load becomes evident!

        \paragraph{Instances trigger}
        By knowing the inner details of the system, setting a QTA on the number of instances can be useful. Here is an example of a fired trigger on the number of instances.

        Even though the QTA requirement isn't being violated, the number of instances fires a trigger, where the user can observe that the system is showing early signs of overload.
        \begin{figure}[H]
            \begin{center}
                \includegraphics[scale=0.5]{img/overload_2/fired_samplea.png}
            \end{center}
            \caption{Snapshot for a fired trigger observing the load of a probe. The trigger fires for observed load $> 150$ in a sampling window of 1 second. Even if the QTA requirement (step function) are not being violated, the system is showing early signs of non-linear behaviour.
            \\
            With knowledge about the system inner workings, smart triggers can be set on load to detect non-linear behaviour.}
        \end{figure}
  

    \section{Detecting slower workers in operators}
    \subsection{First to finish application}
        Next, we provide a synthetic application modeling an application that can be modeled by a first to finish operator. The outcome diagram is what is shown in \cref{fig:op}. Assume a send request to ``the cloud'' that waits for a response or a timeout, it is modeled by an FTF operator \cite{dq-tut}.
       \subsubsection{Using the wrong operator}
            What happens if the wrong operator is chosen to represent the causal relationships between the outcomes? What if the user believes that the system diagram is the one we presented before in \cref{fig:mm1k} (1 in \cref{fig:wrong_od})? The result on the oscilloscope will clearly show that something is wrong.
            \begin{figure}[H]
            \begin{minted}{text}
                ftf = worker_1 -> worker_2; (1)
                ... = f:ftf(worker_1, worker_2); (2)
            \end{minted}
            \caption{Two outcome diagram definitions proposed by the engineer for ftf.}
                \label{fig:wrong_od}
            \end{figure}

        \begin{figure}[H]
                \centering
                \begin{subfigure}{.5\textwidth}
                    \centering
                    \includegraphics[width =0.98\textwidth]{img/bad1.png}
                    \label{fig:bad}
                \end{subfigure}%
                \begin{subfigure}{.5\textwidth}%
                    \centering%
                    \includegraphics[width =0.98\textwidth]{img/gooda.png}%
                    \label{fig:good}%
                \end{subfigure}%
                \caption{\textbf{Left}: FTF plot \textbf{with wrong outcome diagram definition} as shown in the oscilloscope. (1, red) Observed $\Delta$Q in sampling window with its confidence bounds between arrows next to it. (2, red) Calculated $\Delta$Q in sampling window with its confidence bounds between arrows next to it. \\
                \textbf{Right}: FTF plot \textbf{with correct outcome diagram definition}. (Red between arrow) Observed $\Delta$Q and calculated $\Delta$Q confidence bounds overlapping.}
                \label{fig:ftf_osc}%
            \end{figure}%

            On the left in \cref{fig:ftf_osc}, we can observe how the sampling window \textbf{calculated $\Delta$Q} and its confidence bounds (2) are clearly greater than the sampling window \textbf{observed $\Delta$Q} (1) and its confidence bounds. This difference tells us that the proposed outcome diagram does not correctly represent the actual system. On the right, if no dependencies are present and the correct operator (2 in \cref{fig:wrong_od}) is chosen, the two $\Delta$Qs (observed and calculated) will overlap, as shown on the right in \cref{fig:ftf_osc}.

                  \subsubsection{Introducing a slower component}
            Let us introduce a slower worker into the system, we introduce an artificial delay into worker\_2 (about 20ms). If the oscilloscope works correctly, the paradigm operations are sound and no dependencies are present in the system, we should not see any difference in the observed and calculated $\Delta$Qs of the FTF operator. Moreover, the FTF $\Delta$Q could be overlaid on top of the faster worker (\texttt{worker\_1}). In \cref{fig:ftf_w1w2}, this is the case, the FTF operator can be overlaid on top of worker\_1's graph. 

            \begin{figure}[H]
                \centering 
                \begin{subfigure}{.5\textwidth}
                    \centering
                    \includegraphics[width =0.98\textwidth]{img/ftfa.png}
                    \label{fig:ftf_art_d}
                \end{subfigure}%
                \begin{subfigure}{.5\textwidth}%
                    \centering%
                    \includegraphics[width =0.98\textwidth]{img/delay32.png}%
                    \label{fig:ftf_art_dw}%
                \end{subfigure}%
                \caption{\textbf{Left}: ``ftf'' $\Delta$Qs. Sampling window observed and calculated $\Delta$Qs (1) and their confidences bounds (between red arrow) overlapping.
                \textbf{Right}: worker\_1 (1, red) and worker\_2 (2, red) $\Delta$Qs confidence bounds.
                The FTF plot correctly displays how worker\_2 does not have an effect on the FTF plot. The FTF plot can be overlaid on top of worker\_1.}
                \label{fig:ftf_w1w2}%
            \end{figure}%


    \subsection{All to finish application}
    We can extend the previous application to an all-to-finish operator, this operator can for instance parallel work, a task that requires a lot of computation and can be done in separate pieces by separate workers. \cite{dq-tut}

        \subsubsection{Introducing a slower component}
            Like we did for the FTF operator, let's introduce a slower work into the mix. We introduce a slight delay to show how even a few milliseconds can be noticeable right away by a keen eye (or by triggers, which avoids having to look constantly at the graphs). The delay is a 2ms sleep on worker\_2.

        \paragraph{Worker's performance}
            We present here the worker's graphs as shown on the oscilloscope.

            \begin{figure}[H]
                \centering
                \begin{subfigure}{.5\textwidth}
                    \centering
                    \includegraphics[width =0.98\textwidth]{img/atfw1.png}
                    \label{fig:atfw1}
                \end{subfigure}%
                \begin{subfigure}{.5\textwidth}%
                    \centering%
                    \includegraphics[width =0.98\textwidth]{img/atfw2.png}%
                    \label{fig:atfw2}%
                \end{subfigure}%
                \caption{Left: worker\_1 plot. Right: worker\_2 plot.}
                \label{fig:atf}%
            \end{figure}%

            The difference in the worker's $\Delta$Q can be noticed with $\Delta \text{Q}_w2 > \Delta \text{Q}_w1$. The difference can then be observed in the all-to-finish plot, where the operator's $\Delta$Qs (both observed and calculated) can be overlaid on top of worker\_2 $\Delta$Q, showing once again that the $\Delta$QSD foundation is sound.

            \begin{figure}[H]
                \begin{center}
                    \includegraphics[scale = 0.5]{img/atfw1w2.png}
                \end{center}
            \end{figure}

    These plots show the usefulness of $\Delta$QSD, the system can be decomposed to understand which part of the system is showing hazards, furthermore, the causal relationships can be observed to determine the behaviour of a part down to the single component.

