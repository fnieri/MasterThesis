\chapter*{Abstract}
    It is difficult to study the detailed behaviour of large distributed systems while they are running. What happens when there is an overload? How can we feel something is wrong with the system before anything problematic can be observed? Current observability tools do not meet observability requirements when it comes to detecting problems early enough in running systems. 
    
    This thesis aim to provide further proof about how the $\Delta$QSD paradigm can be used to study the behaviour of running systems and to explore tradeoffs in system design, thanks to the implementation of the \textbf{$\Delta$Q oscilloscope}, a real time graphical dashboard that gives insights into a running Erlang system. The development of an Erlang wrapper, named \texttt{dqsd\_otel}, allows the running system to communicate with the oscilloscope to receive real time insights about the execution of the former.
    
    The oscilloscope performs statistical computations on the time series data it receives, thanks to the $\Delta$QSD paradigm. We provide a set of triggers which are set to capture rare events, like an oscilloscope would, and give a snapshot of the system under observation as if it was frozen in time. An implementation of a syntax to create outcome diagrams allows the creation of outcome diagrams which give an "observational view" of the system. Furthemore, the implementation of efficient algorithms allows for the computations to be done rapidly on precise representations of components.

    We first provide an extensive summary of $\Delta$QSD concepts, which have been extended to allow the instrumentation of Erlang systems. Subsequently, we explain the user level concepts which are essential to understand how to use the oscilloscope and understand what is displayed on the screen, delving later on into the mathematical foundations of the concepts. Lastly, we provide synthetic applications which prove the soundness of $\Delta$QSD and show how the oscilloscope is be able to detect problems in a running system, diagnose a system and explore design tradeoffs.


\iffalse
    The thesis aims to develop the $\Delta$Q oscilloscope. The oscilloscope is based a graphical dashboard to observe running Erlang systems in real time. The oscilloscope works by communicating to a wrapper attached to an Erlang system, the System Under Test. By attaching itself to a running system, the wrapper can send information about the execution of components to the oscilloscope.

    To tackle these problems, PNSol Ltd. developed the $\Delta$QSD paradigm a novel metrics-based and quality-centric paradigm that uses formalised outcome diagrams to explore the performance consequences of design decisions, as a performance blueprint of the system. PNSol Ltd. analyses the behaviour of existing systems using $\Delta$QSD, but the technology only works a posteriori, thIn order to use the $\Delta$Q Oscilloscope, the first step is to determine the system's outcome diagram. The system's outcome diagram is a directed graph that shows the causal relationships between the system's internal operations (called outcomes) and its overall inputs and outputs. Each outcome corresponds to a system component. It has a start event and stop event, when the component is called and when it returns.

For each outcome of interest, an observation point (probe) is attached to measure the delay of that outcome. As many probes are put into the system as are needed to observe the desired system behaviours. The probes are implemented on top of OpenTelemetry tracing ability. Each probe sends a time series of data to the $\Delta$Q Oscilloscope, which performs statistical computations on all the time series and displays the results in real time.
\fi
