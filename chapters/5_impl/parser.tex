\section{Parser}
        To parse the system, we use the C++ ANTLR4 (ANother Tool for Language Recognition) library \cite{antlr4}. 
        \subsection{ANTLR}
    According to ANTLR website \cite{antlr4}: 
    \begin{quote}
        ``ANTLR is a parser generator for reading, processing, executing or translating structured text files. ANTLR generates a parser that can build and walk parse trees.''
    \end{quote}

 ANTLR is just one of the many parsers generators available in C++ (flex/bison \cite{flexb}, lex/yacc \cite{lexy}). Although it presents certain limitations, its generated code is simpler to handle and less convoluted with respect to the other possibilities.

        ANTLR uses Adaptive LL(*) \textit{(ALL(*))} parser, namely, it will ``move grammar analysis to parse-time, without the use of static grammar analysis''. \cite{antlr}

        \subsection{Grammar}
            ANTLR provides a yacc-like metalanguage \cite{antlr} to write grammars. The grammar we have written can be found in the appendix \cref{app:grammar}.
              
    \subsubsection{Limitations}
        A previous version was implemented in Lark \cite{lark}, a python parsing toolkit. The python version was quickly discarded due to a more complicated integration between Python and C++. Lark provided Earley(SPPF) strategy which allowed for ambiguities to be resolved, which is not possible in ANTLR. \\
        For example the following system definition presents a few errors:
        \begin{minted}{text}
        probe = s -> a -> f -> p;
        \end{minted}
    While Lark could correctly guess that everything inside was an outcome, ANTLR expects \texttt{``:''} after ``s, a, f'' and ``p'', thus, one can not name an outcome by these characters, as the parser generator thinks that an operator or a probe will be next. 
