\section{Parser}
        To parse the system, we use the C++ ANTLR4 (ANother Tool for Language Recognition) library. 
        \subsection{ANTLR}
    ANTLR is a parser generator for reading, processing, executing or translating structured text files. ANTLR generates a parser that can build and walk parse trees \cite{antlr4}.

 ANTLR is just one of the many parsers generators available in C++ (flex/bison, lex, yacc). Although it presents certain limitations, its generated code is simpler to handle and less convoluted with respect to the other possibilities.

        ANTLR uses Adaptive LL(*) \textit{(ALL(*))} parser, namely, it will move grammar analysis to parse-time, without the use of static grammar analysis. \cite{antlr}

        \subsection{Grammar}
            ANTLR provides a yacc-like metalanguage \cite{antlr} to write grammars. Below, is the grammar for our system:
            \begin{minted}{text} 
grammar DQGrammar;

PROBE_ID: 's';
BEHAVIOR_TYPE: 'f' | 'a' | 'p';
NUMBER: [0-9]+('.'[0-9]+)?;
IDENTIFIER: [a-zA-Z_][a-zA-Z0-9_]*;
WS: [ \t\r\n]+ -> skip;

start: definition* system? EOF;

definition: IDENTIFIER '=' component_chain ';';

component_chain : component ('->' component)*;

component : behaviorComponent | probeComponent | outcome;

behaviorComponent : BEHAVIOR_TYPE ':' IDENTIFIER ('[' probability_list ']')? '(' component_list ')';

probeComponent : PROBE_ID ':' IDENTIFIER;

probability_list: NUMBER (',' NUMBER)+;
component_list: component_chain (',' component_chain)+;
outcome: IDENTIFIER;
\end{minted}
             
    \subsubsection{Limitations}
        A previous version was implemented in Lark \cite{lark}, a python parsing toolkit. The python version was quickly discarded due to a more complicated integration between Python and C++. Lark provided Earley(SPPF) strategy which allowed for ambiguities to be resolved, which is not possible in ANTLR. \\
        For example the following system definition presents a few errors:
        \begin{minted}{text}
        probe = s -> a -> f -> p;
        \end{minted}
    While Lark could correctly guess that everything inside was an outcome, ANTLR expects \texttt{``:''} after ``s, a, f'' and ``p'', thus, one can not name an outcome by these characters, as the parser generator thinks that an operator or a probe will be next.
   
