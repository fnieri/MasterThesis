\section{Triggers}
    Much like an oscilloscope that has a trigger mechanism to capture periodic signals or investigate a transient event \cite{osc-t}, the \textit{$\Delta$Q oscilloscope} has a similar mechanism that can recognise when an observed $\Delta$Q violates certain conditions regarding required behaviour and record snapshots of the system.

    Each time an observed $\Delta$Q is calculated, it is checked against the requirements set by the user. If these requirements are not met, a trigger is fired and a snapshot of the system is saved to be shown to the user. 
    
    \subsection{Snapshot}
    A snapshot of the system gives insights into the system before and after a trigger was fired. It gives the user a \textit{still} of the system, as if it was frozen in time. All the $\Delta$Qs which are observed and calculated during the system's execution are stored away. Then, if no trigger is fired, older $\Delta$Qs are removed. \\
    Otherwise, the oscilloscope keeps recording $\Delta$Qs without removing older ones, to allow the user to look at the state of the system before and after the trigger.
    
