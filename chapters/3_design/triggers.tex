\section{Triggers}
    Much like an oscilloscope that has a trigger mechanism to capture periodic signals or investigate a transient event \cite{osc-t}, the \textit{$\Delta$Q oscilloscope} has a similar mechanism that can recognise when an observed $\Delta$Q violates certain conditions regarding required behaviour and record snapshots of the system.

    Each time an observed $\Delta$Q is calculated, it is checked against the requirements set by the user. If these requirements are not met, a trigger is fired and a snapshot of the system is saved to be shown to the user. 
    
    \subsection{Snapshot}
    A snapshot of the system gives insights into the system before and after a trigger was fired. It gives the user a \textit{still} of the system, as if it was frozen in time. All the $\Delta$Qs which are observed and calculated in a polling window are stored away. Then, if no trigger is fired, older $\Delta$Qs are removed, both from the polling window and from the snapshot. \\
    Instead, if a trigger is fired, the oscilloscope keeps recording $\Delta$Qs without removing older ones, to allow the user to look at the state of the system before and after the trigger. The snapshot keeps recording for 5 seconds, after these 5 seconds the snapshot can be viewed by the user. The polling window size stays fixed, while the snapshot size increases during the recording of one. When the snapshot has finished recording, their windows sizes align.
    
