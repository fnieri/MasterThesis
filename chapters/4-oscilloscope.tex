\chapter{Oscilloscope: User level concepts}
    The following chapter gives insights on the concepts of $\Delta$QSD in the oscilloscope needed by the user to understand how the oscilloscope works.
    \begin{itemize}
        \item We first provide insights into how $\Delta$QSD was implemented in the oscilloscope, the parameters that define a probe's $\Delta$Q, its representation and what can be done with $\Delta$Qs. We show how a probe $\Delta$Q will be shown in the oscilloscope
        \item We then provide a language to write outcome diagrams based on an already existing syntax.
        \item Lastly, we explain how to control the system in the oscilloscope and how to interact with the stub
    \end{itemize}

\section{$\Delta$QSD implementation}

Originally, $\Delta$Q(x) denotes the probability that an outcome occurs in a time $t \le x$, defining then the "intangible mass" of such IRV as $1 - \lim_{x\to\infty} \Delta Q (x)$.
We then extend the original definition to fit real time constraints, needing to calculate $\Delta$Qs continuously.

For a given probe, $\Delta$Q($t_l$, $t_u$, $dMax$) is the probability that the time of series with samples between time $t_l < t_u$, an outcome or probe occurs in time t $\le$ dMax.

\subsection{Internal representation of a $\Delta$Q}
    We provide a $\Delta$Q class to calculate the $\Delta$Q of a probe between a lower time bound $t_l$ and an upper time bound $t_u$.
    
    The $\Delta$Q can be calculated in various ways: 
    
    \paragraph{Observed $\Delta$Q}
    
    The first way is by having $n$ collected outcome instances between $t_l$ and $t_u$, calculating its PDF and then calculating the \textit{empirical cumulative distribution function} (ECDF) based on its PDF. This is called the \textbf{Observed $\Delta$Q}.
    
    \paragraph{Calculated $\Delta$Q}
    
    A $\Delta$Q can also be calculated by performing operations which are the result of outcome expressions on two or more $\Delta$Qs, the notion of outcome instances is then lost between calculations, as the interest shifts towards calculating the resulting PDFs and ECDFs. This is called the \textbf{Calculated $\Delta$Q}.
    
    \subsection{dMax}
        The key concept of $\Delta$QSD is having a maximum delay after which we consider that the execution is failed, this is represented in a prove as $dMax$. The user defines, for each prove the maximum delay its execution can have.

Setting a maximum delay for an prove is not a job that can be done one-off and blindly, it is something that is done with an underlying knowledge of the system inner-workings and must be thoroughly fine tuned during the execution of the system by observing the resulting distributions of the obtained $\Delta$Qs. 

We define in our oscilloscope a formula to dynamically define a maximum delay:
\begin{equation}
    dMax = \Delta_{T} * N  
    \label{eq:dMaxU}
\end{equation}
Where:
\begin{itemize}
    \item $\Delta_{t base}$ represents the base width of a bin, equal to 1ms.
    \item $N$ the number of bins.
\end{itemize}

The user must choose:
\begin{itemize}
    \item $\Delta_T$: $\Delta_T$ is a value which can be set via a slider on the dashboard to control the delay of a probe. 
    \item $N$: We define a range $\lbrack 1, 1000 \rbrack$ for $N$. This is a good enough bound to allow for finer grained bins, or less precision if needed. 
\end{itemize}

Some tradeoffs must though be acknowledged when setting these parameters, a higher number of bins corresponds to a higher number of calculations and space complexity, a lower $dMax$ may correspond to more failures. These are all tradeoffs that must be considered by the system engineer and set accordingly.
    \begin{figure}[H]
        \begin{center}
            \includegraphics[scale = 1.2]{tikz/cdf_dmax.pdf}
        \end{center}
        \caption{$\Delta$Q: $dMax$ = 50ms, the CDF will stay constant when delay $> dMax$}
    \end{figure}

    \subsection{QTA}
        A simplified QTA is defined for probes. We define 4 points for the step function at 25, 50, 75 percentiles and the maximum amount of failures accepted for an observable. An observed $\Delta$Q will calculate that based on the samples collected. 

    \subsection{Operations}
    In a previous section we talked about the possible operations that can be performed on and between $\Delta$Qs, the time complexity of FTF, ATF and PC is trivially $\mathcal{O}(N)$ where N is the number of bins. As to convolution, the naïve way of calculating convolution has a time complexity of $\mathcal{O}(N^2)$, this quickly becomes a problem as soon as the user wants to have a more fine-grained understanding of a component. Below we present two ways to perform convolution.

        \subsubsection{Convolution} 
            Convolution allows calculating the sum of delays of two causally linked $\Delta$Q$_A$ and $\Delta$Q$_B$. 
        \begin{figure}[H]
            \begin{center}
                \includegraphics{tikz/comb_dq_comp.pdf}
            \end{center}
        \end{figure}

\subsubsection{Arithmetical operations}
        We can apply a set of arithmetical operations between $\Delta$Qs ECDFs, and on a $\Delta$Q.
    \paragraph{Scaling (multiplication)} A $\Delta$Q can be scaled w.r.t a constant $0 \le j \le 1$. It is equal to binwise multiplication on ECDF bins.
    \begin{equation}
        \hat{f_r}(i) = \hat{f}(i) \cdot j
        \label{eq:mul_ecdf}
    \end{equation}

    \paragraph{Operations between $\Delta$Qs} 
        Addition, subtraction and multiplication can be done between two $\Delta$Q of equal bin width (but not forcibly of equal length) by calculating the operation between the two ECDFs of the $\Delta$Qs:
        \begin{equation}
            \Delta \text{Q}_{AB}(i) = \hat{f_A}(i) [\cdot, +, -] \hat{f_B}(i)
            \label{eq:op_dq}
        \end{equation}
    \subsection{Confidence bounds}
    To observe the stationarity of a system we must observe a window of $\Delta$Qs of an observable and calculate confidence bounds over said windows. The bounds can be updated dynamically by inserting or removing a $\Delta$Q, this allows us to consider a small window of execution rather than observing the whole execution.
        \begin{figure}[H]
            \begin{center}
                \includegraphics[scale=1.2]{tikz/ci.pdf} 
            \end{center}
            \caption{Upper and lower bounds of the mean of multiple $\Delta$Qs. In a system that behaves linearly, the bounds will be close to the mean, once the overload is approaching, or a system is showing behaviour that diverges from a linear one, the bounds will be larger.}
        \end{figure}

  

\section{$\Delta$Q display}
    An probe's displayed graph must contain the following functions:
    \begin{itemize}
        \item The mean and confidence bounds of a window of previous $\Delta$Qs.
        \item The observed $\Delta$Q($t_l, t_u, dMax$).
        \item If applicable, the calculated $\Delta$Q from the components showing the causal links of a probe.
        \item Its QTA (if defined).
    \end{itemize}
    This allows for the user to observe if a $\Delta$Q has deviated from normal execution, analyse its stationarity, nonlinearity and observe its execution.
    \iffalse 
    \begin{figure}[!ht]    
    \includegraphics[scale =0.8, width=\textwidth]{img/dqdispl.png}
    \end{figure}
    \fi

  \section{Outcome diagram}
        An abstract syntax for outcome expressions has already been defined in a previous paper, nevertheless, the oscilloscope provides additional features not included in the original syntax and, moreover, needs a textual way to define an outcome diagram.  \\
        We define thus a grammar to create an outcome diagram in our oscilloscope, our grammar is a textual interpretation of the abstract syntax. \\
        \subsection{Observables}
            Below is a way to define the observables in a system.
            \subsubsection{Outcome}
                In the definition of the system or of a probe, an outcome is defined with its name
        \begin{minted}{text}
            probe = outcomeName;
        \end{minted}
            \subsubsection{Probes}
                
        A probe can contain one component or a sequence of causally linked components. \\
        The user can define as many probes as they want, they have to be declared as follows:
   \begin{minted}{text}
        probe = component [-> component2];
        probe2 = newComponent -> anotherComponent;
   \end{minted}

    Probes will not be parsed after the system has been defined, in the case below, an error will be thrown.
    \begin{minted}{text}
        probe = ...;
        probe2 = ...;
        system = ...;
        probe3 = ...;
    \end{minted}
    
    Proes can be reused in other probes or in the system by adding a s: before they are used.
    \begin{minted}{text}
        probe3 = s:probe -> s:probe2;
    \end{minted}
 
        \subsection{Operators}
        To build a system, we must define some operators, below is how they can be defined. About first-to-finish, all-to-finish and probabilistic choice, they must contain at least two components, this is because the operations to calculate the DeltaQ of these operators rely on using the CDF of the components that define the operator.

        \subsubsection{Causal link}
            A causal link between two components can be defined by a right arrow from \texttt{component\_i} to \texttt{component\_j}
        \begin{minted}{text}
            component_i -> component_j 
        \end{minted}
        
        \subsubsection{All-to-finish operator}
            An all-to-finish operator needs to be defined as follows:
            \begin{minted}{text}
                a:name(component1, component2...)
            \end{minted}

        \subsubsection{First-to-finish operator}
            A first-to-finish operator needs to be defined as follows.
            \begin{minted}{text}
                f:name(component1, component2...)
            \end{minted} 

        \subsubsection{Probabilistic choice operator}
            A probabilistic choice operator needs to be defined as follows:
            \begin{minted}{text}
                p:name[probability_1, probability_2, ... probability_i](component_1, component_2, ..., component_i) 
            \end{minted}
            In addition to being comma separated, the number of probabilities inside the brackets must match the number of components inside the parentheses. For $n$ probabilites $p_i$, $0 < p_i < 1$, $\sum_{i = 0}^{n} p_i = 1$ 
        \subsection{Limitations}
            Our system has a few limitations compared to the theoretical applications of $\Delta$Q, namely, no cycles are allowed in the definition of a system.
        \begin{minted}{text}
            probe = s:probe_2;
            probe_2 = s:probe;
        \end{minted}
        The above example is not allowed and will raise an error when defined. 
   

\section{Dashboard}
    The dashboard is devised of multiple sections where the user can interact with the oscilloscope, create the system, observe the behaviour of its components, set triggers.

    \subsection{Sidebar}
        The sidebar has multiple tabs, we explain here the responsibility of each one.

    \subsubsection{System/Handle plots tab}

    \paragraph{System creation}
        In this tab the user can create its system using the grammar defined before, he can save the text he used to define the system or load it, the system is saved to a file with any extension, we nevertheless define an extension to save the system to, the extension \texttt{.dq}.
        If the definition of the input is wrong, he will be warned with a pop up giving the error the parser generator encountered in the creation of a system.

    \paragraph{Adding a plot}
        Once the system is defined, the user can choose the probes he wants to plot. They can select multiple probes per plot and display multiple plots on the oscilloscope window.
    
    \paragraph{Polling rate}
        The user can choose the polling rate of the system: How often $\Delta$Qs are calculated and displayed in the oscilloscope.

    \paragraph{Editing a plot}
        By clicking onto a plot that is being shown, the user can choose to add or remove probes to and from it. Multiple probes can be selected to either be removed or added.

    \subsubsection{Parameters tab}
        In this tab, the user can define parameters for the probes they have defined.

    \paragraph{Set a QTA}
        The user is given the choice to set a QTA for a given observable, they have 4 fields where they can fill in which correspond to the percentiles and the maximum amount of failures allowed, they can change this dynamically during execution.

    \paragraph{dMax, bins}
        The user has a slider which goes from -10 to 10, where they can set the parameters we explained previously, $n$, the exponent of $\Delta_{tbase} \cdot 2^n$ and the bins $N$. When these informations are saved by the user, the new $dMax$ is transmitted to the stub and saved for the selected observable.

    \subsubsection{Triggers tab}
        In the triggers tab the user can set triggers and observe the snapshots of the system.

    \paragraph{Set triggers}
        The user can set which triggers to fire for the probes they desire, they are given checkboxes to decide which ones to set as active or not (by default, the triggers are deactivated).
    
    \paragraph{Fired triggers}
        Once a trigger is fired, the system start a timer, during which all probes start recording the observed $\Delta$Qs (and the calculated ones if applicable) without discarding older ones. Once the timer expires, the snapshot is saved for the user in the triggers tab. In the dashboard, it indicates when the trigger was fired (timestamp) and the name of the probe which fired it.
    
    \subsection{Plots window}
        To the left, the main window shows the plots of the probes being updated in real time. 

    \subsection{Stub controls}
        Below the sidebar, two buttons are present, these buttons communicate to the stub. 
         
        The \textbf{start stub} button sends a message to the stub, telling it to start sending spans. The \textbf{stop stub} button stops it.

%\section{Triggers}
    There are two available triggers
    \subsection{Load}
        A trigger on an observed $\Delta$Q can be fired if the amount of outcome instances received in a polling rate is greater than what the user defines:
    \begin{center}
        nSamples($\Delta$Q($t_l, t_u, dMax$)) > maxAllowedInstances 
    \end{center}

    \subsection{QTA}
        A trigger on an observed $\Delta$Q can be fired if:
        \begin{center}
            $\Delta$Q$_{obs}$ $<$ observableQTA
        \end{center}


