\section{Contributions}
    There are a few contributions that make the master thesis and thus, the oscilloscope, possible:
    \begin{itemize}
        \item A graphical interface to display probes' $\Delta$Qs.
        \item An Erlang OpenTelemetry adapter to give OpenTelemetry spans a notion of failure and to communicate with the oscilloscope.
        \item The implementation of $\Delta$QSD concepts from theory to practice, including a textual syntax to create outcome diagrams, derived from the original algebraic syntax.
        \item An efficient convolution algorithm based on the FFTW3 library.
        \item A system of triggers to catch rare events when system behaviour fails to meet quality requirements, creating a snapshot of the system, giving the user insights about their system's behaviour.
        \item Synthetic applications to test the effectiveness of $\Delta$QSD on diagnosing running systems.
    \end{itemize}
    These contributions can show that the $\Delta$QSD paradigm can be translated to real-time observation of running system. Furthemore, it proves that the paradigm has its practical applications and is not limited to a theoretical view of system design.

