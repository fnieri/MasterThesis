\section{Contributions}
    We make the following principal contributions in the master thesis:
    \begin{itemize}
        \item The $\Delta$Q oscilloscope, from design to implementation, to plot real-time $\Delta$Q graphs. The oscilloscope contributions include:
        \begin{itemize}
            \item A graphical interface in Qt.
            \item The underlying implementation of $\Delta$QSD concepts, including a textual syntax to create outcome diagrams derived from the original algebraic syntax.
            \item Efficient convolution algorithms.
            \item A system of triggers to catch rare events, when system behaviour fails to meet quality requirements.
        \end{itemize}
        \item The $\Delta$Q adapter to communicate from the Erlang application to the oscilloscope.
        \item The evaluation of the effectiveness of the oscilloscope on synthetic applications.
        \item The evaluation of the efficiency of the basic operations regarding the oscilloscope: convolution, graphing and the adapter overhead.
    \end{itemize}

    These contributions can show that the $\Delta$QSD paradigm can be translated from a posteriori analysis to real-time observation of running system. Furthermore, it reinforces the validity of the paradigm.

