\section{Objective}
         This project will develop a practical tool, the \textbf{$\Delta$Q oscilloscope}, for the Erlang developer community. 
    
    The Erlang language and Erlang/OTP platform are widely used to develop distributed applications that must perform reliably under high load \cite{erl}. The tool will provide useful information for these applications both for understanding their behaviour, for diagnosing performance issues, and for optimising performance over their lifetime. \cite{post}

    The $\Delta$Q Oscilloscope will perform statistical computations to show real time graphs about the performance of system components. With the oscilloscope prototype we will present in this paper, we are aiming to show that the $\Delta$QSD paradigm is not only a theoretical paradigm, but it can be employed in a real-time tool to diagnose distributed systems. Its application can then be further extended to large systems once the oscilloscope is refined.  
 
    The paradigm ideal target is ``large distributed applications handling many independent tasks where performance and reliability are important''. \cite{dq-tut} This also applies to the oscilloscope.
    

