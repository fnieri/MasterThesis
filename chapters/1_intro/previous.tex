\section{Previous work}
    The $\Delta$QSD paradigm has been formalised across different papers \cite{myo} \cite{art} and was brought to the attention of engineers via tutorials \cite{dq-tut}, and to students at Université Catholique de Louvain. \cite{dq-ucl} 
    
    A Jupyter notebook workbench has been made available on GitHub \cite{dqsd-wkb}. It is meant as an interactive tool to show how the $\Delta$QSD paradigm can be applied to real life examples. It shows real time $\Delta$Q graphs for typical outcome diagrams, but is not adequate to be scaled to real time systems.
    
    Observability tools such as Erlang tracing \cite{erl-t} and OpenTelemetry \cite{otel-e} lack the notions of failure as defined in $\Delta$QSD, which allows detecting performance problems early on. We base our program on OpenTelemetry to incorporate already existing notions of causality and observability to augment their capabilities and make them suitable to work with the $\Delta$QSD paradigm.
