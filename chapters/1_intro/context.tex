\section{Approach}
    In the context of this thesis, the \textbf{$\Delta$QSD paradigm} has been used to develop a tool to study the real-time behaviour of running systems. \\
    According to this tutorial: \cite{dq-tut}
    \begin{quote}
        $\Delta$QSD is an industrial-strength approach for large-scale system design that can predict performance and feasibility early on in the design process.  
    \end{quote}
    The paradigm has been developed over 30 years by the people around \textbf{Predictable Network Solutions Ltd.} \cite{pnsol} It has had various successful uses in the context of distributed and large-scale projects. Moreover, it is the basis of Broadband forum's TR452 standard series, used in instrumenting data networks \cite{dq-br}.

    Thanks to outcome diagrams and statistical representations of component's behaviour, performance and feasibility can be predicted with the paradigm at high load, even if the system is not fully defined.  \cite{dq-tut} \cite{myo}
 
    While the paradigm has been successfully applied in \textbf{a posteriori} analysis, there is no way yet to analyse a distributed system which is running in real time with $\Delta$QSD! This is where the \textbf{$\Delta$Q oscilloscope} comes in.
