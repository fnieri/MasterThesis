\section{Roadmap}
    The following thesis will give the reader everything that is needed to use the oscilloscope and exploit it to its full potential.

    We divided the thesis in multiple chapters, below is the roadmap of the content:
    \begin{itemize}
        \item The background chapter gives the reader an extensive background into the theoretical foundations of $\Delta$QSD, which are the basis of the oscilloscope and are fundamental to understand how to correctly use and analyse the output given by the oscilloscope. Secondly, an introduction to OpenTelemetry, the library we base our Erlang adapter on. Lastly, we provide what we believe are the current limitations of the observability tool and how we plan to tackle them.
        \item The design chapter first provides the "measurement concepts", these concepts serve as an introduction to understand the following chapters and as a bridge from OpenTelemetry to the oscilloscope.  We then delve on how the parts of the system interact together and how to correctly apply the concepts we just introduced in the application side and in the oscilloscope. Lastly, after having introduced the oscilloscope, we explain more abstract concepts implemented in it, like triggers and sliding windows.
        \item We then present the oscilloscope in two different chapters, first providing "user level concepts" of how $\Delta$QSD is used and what the user should expect visually from the dashboard. It also provides a complete explanation on how to write outcome diagrams and what the different sections on the dashboard do.
            Secondly, a more low level explanation, which goes into more technical details of the parts that compose the oscilloscope and the mathematical explanations of $\Delta$QSD concepts explained in the previous chapter.
        \item We then provide synthetic applications which have been tested with the oscilloscope that demonstrate the usefulness of the oscilloscope in a distributed setting. We also perform evaluations of the performance of the different parts we have developed to understand the overhead that are present.
    \end{itemize}

    We end by providing future possibilities which can be explored to improve the application.    In the appendix, we provide a user manual to help users use the oscilloscope, along with C++ and Erlang source code of the oscilloscope and the adapter.
   \sloppy The oscilloscope (\url{https://github.com/fnieri/DeltaQOscilloscope}) and adapter(\url{https://github.com/fnieri/dqsd_otel}) can be found on GitHub as open source projects.
