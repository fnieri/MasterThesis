\section{Roadmap}
    This thesis gives the reader everything that is needed to use the oscilloscope and exploit it to its full potential.

    We divided the thesis in multiple chapters:
    \begin{itemize}
        \item Chapter 2 gives the reader a background of the theoretical foundations of $\Delta$QSD, which are the basis of the oscilloscope and are fundamental to understand what is shown in the oscilloscope. Secondly, an introduction to OpenTelemetry, the framework our Erlang adapter is built on top of. Lastly, we provide what we believe are the current limitations of the observability tool and how we plan to tackle them.
        \item Chapter 3 first provides the ``measurement concepts''. These concepts serve as an introduction to understand the following chapters and as a bridge from OpenTelemetry to the oscilloscope.  We then delve on how the different parts of our design interact together and how to correctly apply the concepts we introduced. Lastly, after having introduced the oscilloscope, we explain abstract concepts implemented in it, like triggers and sliding windows.
        \item Chapter 4 \& 5 present the oscilloscope. First providing ``user level concepts'' of how $\Delta$QSD is used and what the user should expect visually from the dashboard. Chapter 4 also provides a complete explanation on how to write outcome diagrams and what the different sections on the dashboard do.
            Secondly, a more low level explanation, which goes into more technical details of the parts that compose the oscilloscope and the mathematical explanations of $\Delta$QSD concepts explained in the previous chapter.
        \item Chapter 6 provides synthetic applications which have been tested with the oscilloscope that demonstrate the usefulness of the oscilloscope in a distributed setting. In Chapter 7 we perform evaluations of the performance of the different parts we have developed to understand the overhead that are present.
    \end{itemize}

    Chapter 8 provides future possibilities which can be explored to improve the application. In the appendix, we provide a user manual to help users use the oscilloscope, along with C++ and Erlang source code of the oscilloscope and the adapter.
   \sloppy The oscilloscope (\url{https://github.com/fnieri/DeltaQOscilloscope}) and adapter(\url{https://github.com/fnieri/dqsd_otel}) can be found on GitHub as open source projects.
