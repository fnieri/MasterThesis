  \section{Outcome diagram}
        An abstract syntax for outcome expressions and consequently outcome diagrams has already been defined in a previous paper \cite{art}, nevertheless, the oscilloscope provides additional features not included in the original syntax and, moreover, needs a textual way to define an outcome diagram. 
       
        We define thus a grammar to create an outcome diagram in our oscilloscope, our grammar is a textual interpretation of the abstract syntax.
        \subsection{Probes}
            To attach probes in the oscilloscope, the user must define outcomes and probes that observe outcome expressions.
            \subsubsection{Outcome}
                In our system an outcome is defined with its name
                \begin{minted}{text}
                    ... = outcomeName;
                \end{minted}
            Outcomes cannot be defined on their own and must be observed by other probes, we explain below how.
        \subsubsection{Probes containing outcome expressions}
            A probe can contain one outcome or  a sequence of causally linked components.
            The user can define as many probes that contain outcome expressions as they want, they have to be declared as follows:
            \begin{minted}{text}
                probe = component [-> component2];
                probe2 = newComponent -> anotherComponent;
            \end{minted}
    
            These probes can be reused in other probes or in the system by adding \texttt{"s:"} (subsystem) before they are used.
            \begin{minted}{text}
                probe3 = s:probe -> s:probe2;
            \end{minted}
        The lines defining these probes must be semicolon terminated. 
        \subsection{Operators, outcome expressions}
            To build a system, we must define the relations between outcomes and outcome expressions, below is how they can be defined. 

            First-to-finish, all-to-finish and probabilistic choice must contain at least two components, this is because the operations to calculate the \textit{calculated} $\Delta$Q rely on using the CDF of the components that define the operator.
            
        \subsubsection{Causal link}
            A causal link between two components can be defined by a right arrow from \texttt{component\_i} to \texttt{component\_j}
        \begin{minted}{text}
            component_i -> component_j 
        \end{minted}
        
        \subsubsection{All-to-finish operator}
            An all-to-finish operator needs to be defined as follows:
            \begin{minted}{text}
                a:name(component1, component2...)
            \end{minted}

        \subsubsection{First-to-finish operator}
            A first-to-finish operator needs to be defined as follows.
            \begin{minted}{text}
                f:name(component1, component2...)
            \end{minted} 

        \subsubsection{Probabilistic choice operator}
            A probabilistic choice operator needs to be defined as follows:
            \begin{minted}{text}
                p:name[probability_1, probability_2, ... probability_i](component_1, component_2, ..., component_i) 
            \end{minted}
            In addition to being comma separated, the number of probabilities inside the brackets must match the number of components inside the parentheses. For $n$ probabilites $p_i$, $0 < p_i < 1$, $\sum_{i = 0}^{n} p_i = 1$ 
        
            
        \subsection{Limitations}
            Our system has a few limitations compared to the theoretical applications of $\Delta$Q, namely, no cycles are allowed in the definition of a system.
        
        \begin{minted}{text}
            probe = s:probe_2;
            probe_2 = s:probe;
        \end{minted}
        The above example is not allowed and will raise an error when defined.  


