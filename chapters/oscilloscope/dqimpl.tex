\section{$\Delta$QSD implementation}

Originally, $\Delta$Q(x) denotes the probability that an outcome occurs in a time $t \le x$, defining then the "intangible mass" of such IRV as $1 - \lim_{x\to\infty} \Delta Q (x)$.
We then extend the original definition to fit real time constraints, needing to calculate $\Delta$Qs continuously.

For a given probe, $\Delta$Q($t_l$, $t_u$, $dMax$) is the probability that the time of series with $n$ outcome instances with end time $t_l \le t_e \le t_u$, an outcome or probe occurs in time t $\le dMax$.

\subsection{Internal representation of a $\Delta$Q}
    We provide a $\Delta$Q class to calculate the $\Delta$Q of a probe between a lower time bound $t_l$ and an upper time bound $t_u$.
    
    The $\Delta$Q can be calculated in various ways: 
    
    \paragraph{Observed $\Delta$Q}
    
    The first way is by having $n$ collected outcome instances between $t_l$ and $t_u$, calculating its PDF and then calculating the \textit{empirical cumulative distribution function} (ECDF) based on its PDF. This is called the \textbf{Observed $\Delta$Q}.
    
    \paragraph{Calculated $\Delta$Q}
    
    A $\Delta$Q can also be calculated by performing operations which are the result of outcome expressions on two or more $\Delta$Qs, the notion of outcome instances is then lost between calculations, as the interest shifts towards calculating the resulting PDFs and ECDFs. This is called the \textbf{Calculated $\Delta$Q}.
    
    \subsection{dMax}
        The key concept of $\Delta$QSD is having a maximum delay after which we consider that the execution is failed, this is represented in a prove as $dMax$. The user defines, for each prove the maximum delay its execution can have.

Setting a maximum delay for an prove is not a job that can be done one-off and blindly, it is something that is done with an underlying knowledge of the system inner-workings and must be thoroughly fine tuned during the execution of the system by observing the resulting distributions of the obtained $\Delta$Qs. 

We define in our oscilloscope a formula to dynamically define a maximum delay:
\begin{equation}
    dMax = \Delta_{T} * N  
    \label{eq:dMaxU}
\end{equation}
Where $\Delta_{T}$ is the bin width of the $\Delta$Q PDF and ECDF and $N$ their number of bins.

The user must choose both via a slider. $N$ is in the range $\lbrack 1, 1000 \rbrack$. This is a good enough bound to allow for finer grained representation, or less precision if needed. 

Some tradeoffs must though be acknowledged when setting these parameters, a higher number of bins corresponds to a higher number of calculations and space complexity, a lower $dMax$ may correspond to more failures. These are all tradeoffs that must be considered by the system engineer and set accordingly.
    \begin{figure}[H]
        \begin{center}
            \includegraphics[scale = 1]{tikz/cdf_dmax.pdf}
        \end{center}
        \caption{$\Delta$Q: $dMax$ = 50ms, the CDF will stay constant when delay $> dMax$}
    \end{figure}

    \subsubsection{dMax limitation}
        $dMax$ can \textbf{not} be lower than 0 milliseconds and will be rounded to the \textbf{nearest} integer, this is a limitation of Erlang \texttt{send\_after} function which only accepts integers and milliseconds values.

    \subsection{QTA}
        A simplified QTA is defined for probes. We define 4 points for the step function at 25, 50, 75 percentiles and the maximum amount of failures accepted for an observable. An observed $\Delta$Q will calculate that based on the samples collected. 
    \iffalse
    \subsection{Convolution}
    Convolution allows calculating the sum of delays of two causally linked $\Delta$Q$_A$ and $\Delta$Q$_B$. 
        \begin{figure}[H]
            \begin{center}
                \includegraphics{tikz/comb_dq_comp.pdf}
            \end{center}
        \end{figure}
    \fi

    \subsection{Confidence bounds}
    To observe the stationarity of a system we must observe a window of $\Delta$Qs of an observable and calculate confidence bounds over said windows. The bounds can be updated dynamically by inserting or removing a $\Delta$Q, this allows us to consider a small window of execution rather than observing the whole execution.
        \begin{figure}[H]
            \begin{center}
                \includegraphics[scale=1]{tikz/ci.pdf} 
            \end{center}
            \caption{Upper and lower bounds (dashed, red) of the mean (blue) of multiple $\Delta$Qs. In a system that behaves linearly, the bounds will be close to the mean, once the overload is approaching, or a system is showing behaviour that diverges from a linear one, the bounds will be larger.}
        \end{figure}

  
