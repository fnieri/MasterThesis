\section{Triggers}
    Like an oscilloscope that has a trigger mechanism that fires when a signal of interest is recognized by the oscilloscope, our $\Delta$Q oscilloscope has a similar mechanism of triggers that are fired when an observed $\Delta$Q violates certain conditions. Let us define what these triggers are.

    \subsection{Load}
        A trigger on an observed $\Delta$Q can be fired if 
    \begin{center}
        nSamples($\Delta$Q($t_l, t_u, dMax$)) > maxAllowedSamples 
    \end{center}

    \subsection{QTA}
        There are two possible [can change] triggers that can be fired based on the observable defined $\Delta$Q's QTA.   
        \subsubsection{Percentiles}
            A trigger can be fired if:
        \begin{center}
            $\Delta$Q$_{obs}$[percentile] $<$ observableQTA$_{req}$[percentile] \quad $\forall \text{ percentile } \{0.25, 0.5, 0.72\}$
        \end{center}
        \subsubsection{Failure}
            A trigger can be fired on the percentage of failed samples for $\Delta$Q($t_l, t_u, dMax$) if:
        \begin{center}
            success($\Delta$Q$_{obs}$) $<$ success(observableQTA$_{req}$)
        \end{center}

    \subsection{Time series snapshots}
        When a trigger is fired, the oscilloscope will capture ...
