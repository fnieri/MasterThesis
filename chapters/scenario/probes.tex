\section{Probes}

To observe a system, we must put probes in it. For each outcome of interest, a probe (observation point) is attached to measure the delay of the outcome, like one would in a true oscilloscope. 

Consider the figure below, a probe is attached at every component to measure their $\Delta$Qs ($c_2, c_3$),  Another probe ($p_1$) is inserted at the beginning and end of the system to measure the global execution delay. Thanks to this probe, the user can observe the $\Delta$Q \textit{"observed at $p_1$"}, which is the $\Delta$Q which was calculated from the data received by inserting probe $p_1$. The \textit{$\Delta$Q "calculated at $p_1$"} is the resulting $\Delta$Q from the convolution of the observed $\Delta$Qs at $c_2$ and $c_3$.   
    \begin{figure}[H]
        \begin{center}
            \includegraphics[scale=1.8]{tikz/probes.pdf}
        \end{center}
        \caption{Probes inserted in a component diagram. In an applications instrumented with OpenTelemetry, $p_1$ could be considered the root span, $c_1$ and $c_2$ its children spans sharing a causal link.}
        \label{fig:probes}
    \end{figure}



