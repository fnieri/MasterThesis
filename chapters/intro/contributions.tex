\section{Contributions}
    There are a few contributions that make the master thesis and thus, the oscilloscope, possible:
    \begin{itemize}
        \item A graphical interface to display $\Delta$Q plots for outcomes.
        \item An Erlang OpenTelemetry wrapper to give OpenTelemetry spans a notion of failure and to communicate with the oscilloscope.
        \item An implementation of a syntax, derived from the original algebraic, syntax to create outcome diagrams. 
        \item The implementation of $\Delta$QSD concepts from theory to practice, allowing outcomes and probes to be displayed and analysed on the oscilloscope.
        \item An efficient convolution algorithm based on the FFTW3 library.
        \item A system of triggers to catch rare events when system behaviour fails to meet quality requirements.
        \item Synthetic applications to test the effectiveness of $\Delta$QSD on diagnosing systems and their feasibility.
    \end{itemize}
    These contributions can show that the $\Delta$QSD has its practical applications and is not limited to a theoretical view of system design

