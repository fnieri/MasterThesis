\section{Context}
    $\Delta$QSD is an industrial-strength approach for large-scale system design that can predict performance and feasibility early on in the design process.  
    Developed over 30 years by a small group of people around Predictable Network Solutions Ltd, the paradigm has been applied in various industrial-scale problems with huge success and large savings in costs. \cite{dq-tut} Moreover, it is the basis of Broadband forum's TR452 standard series, used in instrumenting data networks. \cite{dq-br}

    Modern software development practices successfully fail to adequately consider essential quality requirements or even to consider properly whether a system can actually meet its intended outcomes, particularly when deployed at scale, the $\Delta$QSD paradigm addresses this problem! \cite{art}

       $\Delta$QSD has important properties which make its application to distributed projects interesting, it supports:
    \begin{itemize}
        \item A compositional approach that considers performance and failure as first-class citizens. 
        \item Stochastic approach to capture uncertainty throughout the design approach.
        \item Performance and feasibility can be predicted at high system load for partially defined systems.
    \end{itemize}
    
    While the paradigm has been successfully applied in \textbf{a posteriori} analysis, there is no way yet to analyse a distributed system which is running in real time with $\Delta$QSD! This is where the $\Delta$Q oscilloscope comes in. 
