\section{Roadmap}
    The following thesis will give the reader everything that is needed to use the Oscilloscope and exploit it to its full potential.

    We divided the thesis in multiple chapters, below is the roadmap of the content:
    \begin{itemize}
        \item The background chapter gives the reader an introduction to the tools we leverage in our program, namely, OpenTelemetry and an extensive background into the theoretical foundations of $\Delta$QSD, which are the basis of the oscilloscope and are fundamental to understand how to correctly use and analyse the output given by the oscilloscope.
        \item The design chapter delves into how the parts of the system interact together.
        \item The implementation part is split in two. First, we provide a high level abstraction of how $\Delta$QSD is implemented in the oscilloscope. Secondly, a more low level explanation, which goes into more technical details of the parts that compose the oscilloscope.
        \item Lastly, we provide synthetic applications which have been tested with the oscilloscope and an evaluation of the performance of the oscilloscope.

    \end{itemize}

    We end by providing future possibilities which can be explored, and concepts which we believe ought to be implemented in observabilities tools. 
    In the appendix, we provide a user manual to help users use the oscilloscope, along with Erlang and C++ source code of the oscilloscope and the wrapper.
