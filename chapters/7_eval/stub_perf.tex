\section{$\Delta$Q adapter performance}
    We evaluated the performance of the adapter to measure its impact in a distributed application. We tested the following calls which represent a normal usage of the adapter.
    \begin{itemize}
        \item \texttt{start\_span} $\rightarrow$ \texttt{end\_span}.
        \item \texttt{with\_span} with the following function: \texttt{fun() $\rightarrow$ ok.}
        \item \texttt{start\_span} $\rightarrow$ \texttt{fail\_span}.
    \end{itemize}

    We ran the simulation for 25000 subsequent iterations, the results are shown in \cref{fig:stub_perf}. 

    The overhead is minimal, around 10 microseconds on average to start and end/fail a span. The same cannot be said about with span, the increased overhead is nevertheless due to a function needing to be called inside it for it to record a span.  
    This shows that the adapter can be integrated seamlessly into an application that was instrumented with OpenTelemetry without presenting noticeable overhead.

   \begin{figure}[H]
        \begin{center}
            \includegraphics[width=0.7\linewidth]{img/adapter.pdf}
        \end{center}
        \caption{Adapter performance evaluation.}
        \label{fig:stub_perf}
    \end{figure}

