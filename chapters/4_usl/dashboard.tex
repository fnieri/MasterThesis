\section{Dashboard}
    The dashboard is devised of multiple sections where the user can interact with the oscilloscope, create the system, observe the behaviour of its components, set triggers.

    \subsection{Sidebar}
        The sidebar has multiple tabs, we explain here the responsibility of each one.

    \subsubsection{System/Handle plots tab}
        In this tab, once can create the outcome diagram, add plots and modify current plots. A photo of the tab can be found in \cref{app:sidetab_ex}.
    \paragraph{System creation}
        In this tab the user can create its system with the outcome diagram grammar. They can save the outcome diagram text or load it, the outcome diagram definition is saved to a file with any extension, we nevertheless define an extension to save the system to, the extension \texttt{.dq}.
        If the definition of the input is wrong, they will be warned with a pop-up giving the error the parser generator encountered in the creation of a system.

    \paragraph{Adding a plot}
        Once the system is defined, the user can choose the probes they wants to plot. They can select multiple probes per plot and display multiple plots on the oscilloscope window.
    
    \paragraph{Sampling rate}
        The user can choose the sampling rate of the system: How often $\Delta$Qs are calculated and displayed in the oscilloscope.

    \paragraph{Editing a plot}
        By clicking onto a plot that is being shown, the user can choose to add or remove probes to and from it, thanks to the widget in the lower right corner. Multiple probes can be selected to either be removed or added.

    \subsubsection{Parameters tab}
        In this tab, the user can define parameters for the probes they have defined. A photo of this tab can be found in \cref{app:param_tab}.

    \paragraph{Set a QTA}
        The user is given the choice to set a QTA for a given probe, they have 4 fields where they can fill in which correspond to the percentiles and the maximum amount of failures allowed, they can change this dynamically during execution.

    \paragraph{dMax, bins}
        The user can set the parameters we explained previously, $\Delta$t and $N$. When this information is saved by the user, the new $dMax$ is transmitted to the adapter and saved for the selected observable.

    \subsubsection{Triggers tab}
        In the triggers tab the user can set triggers and observe the snapshots of the system. A photo of the tab can be found in \cref{app:trig_tab}

    \paragraph{Set triggers}
        The user can set which triggers to fire for the probes they desire, they are given checkboxes to decide which ones to set as active or not (by default, the triggers are deactivated).
    
    \paragraph{Fired triggers}
        Once a trigger is fired, the oscilloscope starts a timer, during which all probes start recording the observed $\Delta$Qs (and the calculated ones if applicable) without discarding older ones. Once the timer expires, the snapshot is saved for the user in the triggers tab. In the dashboard, it indicates when the trigger was fired (timestamp) and the name of the probe which fired it.
    
    \paragraph{Snapshot window}
        Clicking on a snapshot, a new window opens. The user can explore a frozen state of the system, being able to explore all the $\Delta$Qs saved in a snapshot. A screenshot of the snapshot window is provided in \cref{app:snapshot}
   \subsubsection{Connection controls}
    
        Here, the user can connect to the Erlang endpoint. They can also start the oscilloscope server to receive outcome instances from the adapter. An image of the tab can be found in \ref{app:con_control}.

        \paragraph{Erlang controls}
            The user can set the IP and the port where the $\Delta$Q adapter is listening from. Two additional buttons communicate with the adapter by sending messages, they can start and stop the adapter's sending of outcome instances.
        
        \paragraph{C++ server controls}
            The user can set the IP and the port for the oscilloscope's server.

\subsection{Plots window}
        To the left, the main window shows the plots of the probes being updated in real time. 
